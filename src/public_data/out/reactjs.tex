\documentclass{article}
% --- PACKAGES ---
\usepackage{csquotes}
\usepackage{hyperref}
\usepackage[a4paper, margin=2cm]{geometry}
\usepackage{listings}
\usepackage[utf8]{inputenc}\usepackage{xcolor}
\usepackage{enumitem}
% --- HYPERREF CONFIG ---
\hypersetup{
    colorlinks=true,
    linkcolor=blue,
    filecolor=blue
    urlcolor=blue,
}
% --- CODEBLOCK CONFIG ---
\definecolor{bgcolor}{rgb}{0.95, 0.95, 0.95}
\lstdefinestyle{cbstyle}{
                backgroundcolor=\color{bgcolor},
                basicstyle=\ttfamily
            }
\lstset{style=cbstyle}
\begin{document}
{\noindent \Huge \textbf{reactjs.org}}\\\\
\\
This repo contains the source code and documentation powering \href{https://reactjs.org/}{reactjs.org}.\\
\\
{\noindent \LARGE \textbf{Getting started}}\\\\
\\
{\noindent \Large \textbf{Prerequisites}}\\\\
\begin{enumerate}[label=\arabic*.]
	\item Git
	\item Node: any 12.x version starting with v12.0.0 or greater
	\item Yarn v1: See \href{https://yarnpkg.com/lang/en/docs/install/}{Yarn website for installation instructions}
	\item A fork of the repo (for any contributions)
	\item A clone of the \href{https://github.com/reactjs/reactjs.org}{reactjs.org repo} on your local machine
\end{enumerate}
{\noindent \Large \textbf{Installation}}\\\\
\begin{enumerate}[label=\arabic*.]
	\item \verb|cd reactjs.org| to go into the project root
	\item \verb|yarn| to install the website's npm dependencies
\end{enumerate}
{\noindent \Large \textbf{Running locally}}\\\\
\begin{enumerate}[label=\arabic*.]
	\item \verb|yarn dev| to start the hot-reloading development server (powered by \href{https://www.gatsbyjs.org}{Gatsby})
	\item \verb|open http://localhost:8000| to open the site in your favorite browser
\end{enumerate}
{\noindent \LARGE \textbf{Contributing}}\\\\
\\
{\noindent \Large \textbf{Guidelines}}\\\\
\\
The documentation is divided into several sections with a different tone and purpose. If you plan to write more than a few sentences, you might find it helpful to get familiar with the \href{https://github.com/reactjs/reactjs.org/blob/main/CONTRIBUTING.md#guidelines-for-text}{contributing guidelines} for the appropriate sections.\\
\\
{\noindent \Large \textbf{Create a branch}}\\\\
\begin{enumerate}[label=\arabic*.]
	\item \verb|git checkout main| from any folder in your local \verb|reactjs.org| repository
	\item \verb|git pull origin main| to ensure you have the latest main code
	\item \verb|git checkout -b the-name-of-my-branch| (replacing \verb|the-name-of-my-branch| with a suitable name) to create a branch
\end{enumerate}
{\noindent \Large \textbf{Make the change}}\\\\
\begin{enumerate}[label=\arabic*.]
	\item Follow the \href{#running-locally}{"Running locally"} instructions
	\item Save the files and check in the browser
	\begin{enumerate}[label=\arabic*.]
		\item Changes to React components in \verb|src| will hot-reload
		\item Changes to markdown files in \verb|content| will hot-reload
		\item If working with plugins, you may need to remove the \verb|.cache| directory and restart the server
	\end{enumerate}
\end{enumerate}
{\noindent \Large \textbf{Test the change}}\\\\
\begin{enumerate}[label=\arabic*.]
	\item If possible, test any visual changes in all latest versions of common browsers, on both desktop and mobile.
	\item Run \verb|yarn check-all| from the project root. (This will run Prettier, ESLint, and Flow.)
\end{enumerate}
{\noindent \Large \textbf{Push it}}\\\\
\begin{enumerate}[label=\arabic*.]
	\item \verb|git add -A && git commit -m "My message"| (replacing \verb|My message| with a commit message, such as \verb|Fix header logo on Android|) to stage and commit your changes
	\item \verb|git push my-fork-name the-name-of-my-branch|
	\item Go to the \href{https://github.com/reactjs/reactjs.org}{reactjs.org repo} and you should see recently pushed branches.
	\item Follow GitHub's instructions.
	\item If possible, include screenshots of visual changes. A preview build is triggered after your changes are pushed to GitHub.
\end{enumerate}
{\noindent \LARGE \textbf{Translation}}\\\\
\\
If you are interested in translating \verb|reactjs.org|, please see the current translation efforts at \href{https://translations.reactjs.org/}{translations.reactjs.org}.\\
\\
\\
If your language does not have a translation and you would like to create one, please follow the instructions at \href{https://github.com/reactjs/reactjs.org-translation#translating-reactjsorg}{reactjs.org Translations}.\\
\\
{\noindent \LARGE \textbf{Troubleshooting}}\\\\
\begin{itemize}
	\item \verb|yarn reset| to clear the local cache
\end{itemize}
{\noindent \LARGE \textbf{License}}\\\\
Content submitted to \href{https://reactjs.org/}{reactjs.org} is CC-BY-4.0 licensed, as found in the \href{https://github.com/open-source-explorer/reactjs.org/blob/master/LICENSE-DOCS.md}{LICENSE-DOCS.md} file.\\
\end{document}
