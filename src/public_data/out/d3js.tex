\documentclass{article}
% --- PACKAGES ---
\usepackage{csquotes}
\usepackage{hyperref}
\usepackage[a4paper, margin=2cm]{geometry}
\usepackage{listings}
\usepackage[utf8]{inputenc}\usepackage{xcolor}
\usepackage{enumitem}
% --- HYPERREF CONFIG ---
\hypersetup{
    colorlinks=true,
    linkcolor=blue,
    filecolor=blue
    urlcolor=blue,
}
% --- CODEBLOCK CONFIG ---
\definecolor{bgcolor}{rgb}{0.95, 0.95, 0.95}
\lstdefinestyle{cbstyle}{
                backgroundcolor=\color{bgcolor},
                basicstyle=\ttfamily
            }
\lstset{style=cbstyle}
\begin{document}
{\noindent \Huge \textbf{D3: Data-Driven Documents}}\\\\
\\
\textbf{D3} (or \textbf{D3.js}) is a JavaScript library for visualizing data using web standards. D3 helps you bring data to life using SVG, Canvas and HTML. D3 combines powerful visualization and interaction techniques with a data-driven approach to DOM manipulation, giving you the full capabilities of modern browsers and the freedom to design the right visual interface for your data.\\
\\
{\noindent \LARGE \textbf{Resources}}\\\\
\begin{itemize}
	\item \href{https://observablehq.com/@d3/learn-d3}{Introduction}
	\item \href{https://github.com/d3/d3/blob/main/API.md}{API Reference}
	\item \href{https://github.com/d3/d3/releases}{Releases}
	\item \href{https://observablehq.com/@d3/gallery}{Examples}
	\item \href{https://github.com/d3/d3/wiki}{Wiki}
\end{itemize}
\\
{\noindent \LARGE \textbf{Installing}}\\\\
\\
If you use npm, \verb|npm install d3|. You can also download the \href{https://github.com/d3/d3/releases/latest}{latest release on GitHub}. For vanilla HTML in modern browsers, import D3 from Skypack:\\
\\
\begin{lstlisting}
<script type="module">

import * as d3 from "https://cdn.skypack.dev/d3@7";

const div = d3.selectAll("div");

</script>
\end{lstlisting}
\\
For legacy environments, you can load D3’s UMD bundle from an npm-based CDN such as jsDelivr; a \verb|d3| global is exported:\\
\\
\begin{lstlisting}
<script src="https://cdn.jsdelivr.net/npm/d3@7"></script>
<script>

const div = d3.selectAll("div");

</script>
\end{lstlisting}
\\
You can also use the standalone D3 microlibraries. For example, \href{https://github.com/d3/d3-selection}{d3-selection}:\\
\\
\begin{lstlisting}
<script type="module">

import {selectAll} from "https://cdn.skypack.dev/d3-selection@3";

const div = selectAll("div");

</script>
\end{lstlisting}
\\
D3 is written using \href{http://www.2ality.com/2014/09/es6-modules-final.html}{ES2015 modules}. Create a custom bundle using Rollup, Webpack, or your preferred bundler. To import D3 into an ES2015 application, either import specific symbols from specific D3 modules:\\
\\
\begin{lstlisting}
import {scaleLinear} from "d3-scale";
\end{lstlisting}
\\
Or import everything into a namespace (here, \verb|d3|):\\
\\
\begin{lstlisting}
import * as d3 from "d3";
\end{lstlisting}
\\
Or using dynamic import:\\
\\
\begin{lstlisting}
const d3 = await import("d3");
\end{lstlisting}
\\
You can also import individual modules and combine them into a \verb|d3| object using \href{https://developer.mozilla.org/en-US/docs/Web/JavaScript/Reference/Global_Objects/Object/assign}{Object.assign}:\\
\\
\begin{lstlisting}
const d3 = await Promise.all([
  import("d3-format"),
  import("d3-geo"),
  import("d3-geo-projection")
]).then(d3 => Object.assign({}, ...d3));
\end{lstlisting}
\end{document}
