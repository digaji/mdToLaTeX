\documentclass{article}
% --- PACKAGES ---
\usepackage{csquotes}
\usepackage{hyperref}
\usepackage[a4paper, margin=2cm]{geometry}
\usepackage{listings}
\usepackage[utf8]{inputenc}\usepackage{xcolor}
\usepackage{enumitem}
% --- HYPERREF CONFIG ---
\hypersetup{
    colorlinks=true,
    linkcolor=blue,
    filecolor=blue
    urlcolor=blue,
}
% --- CODEBLOCK CONFIG ---
\definecolor{bgcolor}{rgb}{0.95, 0.95, 0.95}
\lstdefinestyle{cbstyle}{
                backgroundcolor=\color{bgcolor},
                basicstyle=\ttfamily
            }
\lstset{style=cbstyle}
\begin{document}
{\noindent \LARGE \textbf{Download}}\\\\
\\
Binaries, installers, and source tarballs are available at https://nodejs.org/en/download/.\\
\\
{\noindent \Large \textbf{Current and LTS releases}}\\\\
\\
https://nodejs.org/download/release/\\
\\
The \href{https://nodejs.org/download/release/latest/}{latest} directory is an alias for the latest Current release. The latest-\textit{codename} directory is an alias for the latest release from an LTS line. For example, the \href{https://nodejs.org/download/release/latest-fermium/}{latest-fermium} directory contains the latest Fermium (Node.js 14) release.\\
\\
{\noindent \Large \textbf{Nightly releases}}\\\\
\\
https://nodejs.org/download/nightly/\\
\\
Each directory name and filename contains a date (in UTC) and the commit SHA at the HEAD of the release.\\
\\
{\noindent \Large \textbf{API documentation}}\\\\
\\
Documentation for the latest Current release is at https://nodejs.org/api/. Version-specific documentation is available in each release directory in the \textit{docs} subdirectory. Version-specific documentation is also at https://nodejs.org/download/docs/.\\
\\
{\noindent \LARGE \textbf{Verifying binaries}}\\\\
\\
Download directories contain a \verb|SHASUMS256.txt| file with SHA checksums for the files.\\
\\
To download \verb|SHASUMS256.txt| using \verb|curl|:\\
\\
\begin{lstlisting}
$ curl -O https://nodejs.org/dist/vx.y.z/SHASUMS256.txt
\end{lstlisting}
\\
To check that a downloaded file matches the checksum, run it through \verb|sha256sum| with a command such as:\\
\\
\begin{lstlisting}
$ grep node-vx.y.z.tar.gz SHASUMS256.txt | sha256sum -c -
\end{lstlisting}
\\
For Current and LTS, the GPG detached signature of \verb|SHASUMS256.txt| is in \verb|SHASUMS256.txt.sig|. You can use it with \verb|gpg| to verify the integrity of \verb|SHASUMS256.txt|. You will first need to import \href{#release-keys}{the GPG keys of individuals authorized to create releases}. To import the keys:\\
\\
\begin{lstlisting}
$ gpg --keyserver hkps://keys.openpgp.org --recv-keys DD8F233
\end{lstlisting}
\\
See \href{#release-keys}{Release keys} for a script to import active release keys.\\
\\
Next, download the \verb|SHASUMS256.txt.sig| for the release:\\
\\
\begin{lstlisting}
$ curl -O https://nodejs.org/dist/vx.y.z/SHASUMS256.txt.sig
\end{lstlisting}
\\
Then use \verb|gpg --verify SHASUMS256.txt.sig SHASUMS256.txt| to verify the file's signature.\\
\\
{\noindent \LARGE \textbf{Building Node.js}}\\\\
\\
See \href{BUILDING.md}{BUILDING.md} for instructions on how to build Node.js from source and a list of supported platforms.\\
\\
{\noindent \LARGE \textbf{Security}}\\\\
\\
For information on reporting security vulnerabilities in Node.js, see \href{./SECURITY.md}{SECURITY.md}.\\
\end{document}
